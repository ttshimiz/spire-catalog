Selection of AGN by ultra-hard X-rays provides multiple advantages over other wavelengths. Due to their high energy, ultra-hard X-rays easily pass through any gas or dust in the line of sight providing a direct view of the AGN. Using optical or mid-infrared selection can be problematic due to contamination by the host galaxy. Also ultra-hard X-rays are unaffected by any type of absorption by material obscuring the AGN provided it is optically thin to Compton scattering ($N_{\rm{H}} \lesssim 10^{24}\,\, \rm{cm}^{-2}$) which is a concern for hard X-rays in the 2-10 keV energy range.