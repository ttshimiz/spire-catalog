\subsubsection{Correlation With Ultra-Hard X-ray Luminosity}
Ultra-hard X-ray luminosity directly probes the current strength of the AGN because it likely originates very close to the SMBH. The 14--195 keV luminosity then provides an unambiguous measure of the AGN power especially for Compton-thin sources. If we want to determine whether the AGN contributes in any way to the FIR luminosity, the first check would be to correlate the 14--195 keV luminosity with the each waveband's luminosity. \citet{Melendez:2014yu} ran correlation tests for the PACS wavebands finding a weak, but statistically significant correlation between the 70 and 160 \um{} luminosity and the 14--195 keV luminosity for Sy 1's but not for Sy 2's.

Using the same methods as we did to measure strengths of the correlations between each \textit{Herschel} luminosity, we measured the correlation between each SPIRE luminosity and 14--195 keV luminosity. The last lines of each section of Table~\ref{tab:wave_corrs} lists the results of the correlation tests and Figure~\ref{fig:lum_spire_bat} plots the correlations with gray arrows indicating upper limits.

For the AGN sample as a whole, no significant correlation exists between the SPIRE and 14--195 keV luminosity. All of the $\rho_{\tau}$, after accounting for the partial correlation with distance, are below 0.1 with $P_{\tau}$ either very near or above 5\%. However, when we break the sample up into Sy 1's and Sy 2's and redo the correlation tests, we do find a weak correlation between the 250 and 350 \um{} luminosity and ultra-hard X-ray luminosity for Sy 1's only. Sy 2's $\rho_{\tau}$ are consistent with 0 with $P_{\tau}>70\%$. Visually this can be seen in Figure~\ref{fig:lum_spire_bat} with Sy 1's (top row) showing an upward trend between the SPIRE luminosities and 14--195 keV luminosity (i.e. $L_{\rm BAT}$) whereas the Sy 2's (bottom row) are widely spread out with no clear trend. 

This follows the trend seen in \citet{Melendez:2014yu} where the 70 and 160 \um{} luminosities were correlated much better with 14--195 keV luminosity for Sy 1's than Sy 2's. It also follows the trend where longer wavelength emission displays less and less of a correlation with ultra-hard X-ray luminosity. Previous work in the mid-infrared showed strong correlations between 9, 12, and 18 \um{} emission \citep{Gandhi:2009kx, Matsuta:2012gf, Ichikawa:2012ul} and X-ray luminosity. The correlations begin to disappear when moving to longer wavelengths as seen previously at 90 \um{} in \citet{Ichikawa:2012ul}. \citet{Melendez:2014yu} and this work have extended this analysis to even longer wavelengths at 70, 160, 250, 350, and 500 \um{} showing an ever-decreasing AGN influence that completely disappears at 70 \um{} for Sy 2's and 500 \um{} for Sy 1's.