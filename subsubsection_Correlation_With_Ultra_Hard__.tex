\subsubsection{Correlation With Ultra-Hard X-ray Luminosity}
Ultra-hard X-ray luminosity directly probes the current strength of the AGN because it likely originates very close to the SMBH. The 14--195 keV luminosity then provides an unambiguous measure of the AGN power especially for Compton-thin sources. If we want to determine whether the AGN contributes in any way to the FIR luminosity, the first check would be to correlate the 14--195 keV luminosity with the each waveband's luminosity. \citet{Melendez:2014yu} ran correlation tests for the PACS wavebands finding a weak, but statistically significant correlation between the 70 and 160 \um{} luminosity and the 14--195 keV luminosity for Sy 1's but not for Sy 2's.

Using the same methods as we did to measure strengths of the correlations between each \textit{Herschel} luminosity, we measured the correlation between each SPIRE and 14--195 keV luminosity. The last lines of each section of Table~\ref{tab:wave_corrs} lists the results of the correlation tests and Figure~\ref{fig:lum_spire_BAT} plots the correlations with gray arrows indicating upper limits.

For the AGN sample as a whole, no significant correlation exists between the SPIRE and 14--195 keV luminosity. All of the $\rho_{\tau}$, after accounting for the partial correlation with distance, are below 0.1 with $P_{\tau}$ either at or above 5\%. However, when we break the sample up into Sy 1's and Sy 2's and redo the correlation tests, we find a very weak correlation between the 250 and 350 \um{} luminosity and ultra-hard X-ray luminosity for Sy 1's only ($\rho_{\tau}\sim0.13$). Sy 2's $\rho_{\tau}$ are consistent with no correlation with $P_{\tau}>75\%$. Visually this can be seen in Figure~\ref{fig:lum_spire_BAT} with Sy 1's (left column) showing an upward trend between the SPIRE luminosities and 14--195 keV luminosity (i.e. $L_{\rm BAT}$) whereas the Sy 2's (bottom row) are widely spread out with no clear trend.

We note that while all of the correlation coefficients between the \textit{Herschel} luminosities and ultra-hard X-ray luminosity are statistically significant ($>3\sigma$ away from 0), they are still quite weak with none having a value greater than 0.25. This means that the bulk of the FIR emission originates from processes completely unrelated to the intrinsic strength of the AGN, most likely warm and cold dust near star-forming regions and in the ISM.
  