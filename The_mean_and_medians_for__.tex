The mean and medians for Sy 1's and Sy 2's are very similar, only deviating by at most 0.3 dex. We test for differences between the two samples using the Peto \& Prentice Generalized Wilcoxon test, which is similar to the standard Kolmogorov-Smirnov test but allows for censoring (i.e. upper limits). The test indicates that the probability that Sy 1's and Sy 2's are drawn from the same parent population is 3\%, 3\%, and 9\% for 250, 350, and 500 \um{} respectively. The usual cutoff for significant differences between two samples is 5\%, however our samples are large (149 Sy 1 and 157 Sy 2). Therefore we consider the luminosities to be marginally different at 250 and 350 \um{} and statistically the same at 500 \um. Assuming all of the long wavelength emission is produced by cold dust, this shows that the dust mass in both Sy 1's and Sy 2's are similar. 