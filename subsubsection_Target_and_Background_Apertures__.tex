\subsubsection{Target and Background Apertures}
After the SPIRE source that is associated with the BAT source is found, we used the measured size and shape from \texttt{segment_properties} to construct a target and background aperture. The target aperture is an ellipse and the background aperture is an elliptical annulus. The semimajor and semiminor radii of the target aperture are calculated as 3.5 times the semimajor and semiminor sigma values from \texttt{segment_properties}. The sigma values are measured from the second-order central moments of the detected source and represent the standard deviations along each axis of a 2D Gaussian that has the same second-order moments. The central position of the target aperture is the centroid of the source and the orientation is the same as the measured orientation.

The background annulus has the same central position and orientation as the target aperture. For the inner radius, we increase the semimajor and semiminor axis of the target aperture by 3 pixels. The outer radius is then 1.5 times the inner radius.

In addition to the background annulus, we also construct a series of circular apertures that encircle the target aperture. These have a size of 22", 30", and 42", the recommended size of an aperture for measuring the flux of a point source in the 250, 350, and 500 \um{} maps respectively. While the background annulus is used to measured the local background level, these circular apertures are used to measure the background noise. The SPIRE DRG recommends using local background apertures for the calculation of the background noise because calculating the RMS within the background annulus will underestimate the noise. We construct as many apertures as can fit just outside the target aperture without overlapping but impose a minimum of 6 apertures. Figure~\ref{fig:example_photometry} shows an example of the apertures used in calculating the photometry as well as the segmentation image that was used to find the source and determine its properties to construct the apertures. 

One exception to all of this occurs for small sources. If the constructed target aperture has a semimajor axis smaller than 22", 30", or 42" for 250, 350, and 500 \um{} maps, then we use a circular aperture with these radii. This indicates the source is likely a point source that was either missed using the results from the timeline fitting (Section~\ref{timeline}) or is extended at other SPIRE wavelengths which automatically identifies it as extended at all wavelengths. For these aperture photometry point sources, the background annulus used has a 60" inner radius and a 90" outer radius, the recommended annulus for point source photometry from the SPIRE DRG. The circular background apertures are still constructed in the same way as for extended sources.

The other exception is for sources that live in maps dominated by foreground cirrus emission. Cirrus emission comes from cold dust in the Milky Way galaxy that is along our line of sight to the AGN. It is identified as bright smooth patches that occur over large spatial scales. We visually identified 25 sources that are likely contaminated by Milky Way cirrus. We used point source apertures for the photometry, however instead of placing the background annulus 60--90" away, we placed it right outside the target aperture to get a more accurate estimate of the local background. 