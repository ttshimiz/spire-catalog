\section{Conclusions}
We have produced the \herschel/SPIRE maps for 313 AGN selected from the \swift/BAT 58 month catalog in three wavebands: 250, 350, and 500 \um. Combined with the PACS photometry from \citet{Melendez:2014yu}, the SPIRE flux densities presented in this Paper form the most complete FIR SEDs for a large, nearby, and relatively unbiased sample of AGN. We used two methods for measuring the flux densities: timeline fitting for point sources and aperture photometry for extended and undetected sources. We summarize below the results of our statistical analysis and comparison to the \herschel{} Reference Survey sample of normal star-forming galaxies.
\begin{itemize}
\item While Sy 2s are detected at a higher rate than Sy 1s, after accounting for upper limits, Sy 1s and Sy 2s have identical luminosity distributions at each SPIRE waveband. This indicates that on average, the global cold dust properties of AGN are independent of orientation.

\item Using a partial correlation survival analysis to account for the luminosity-distance effect and upper limits, we find all of the \herschel{} luminosities are correlated with each other suggesting the process (or processes) producing the emission from 70--500 \um{} is connected. Luminosities with the smallest wavelength difference (i.e. 160 and 250 \um) are much more correlated than pairs further apart (i.e. 70 and 500 \um), in agreement with different temperature components associated with different wavebands. This could also point to the AGN affecting the shorter wavebands more than the longer ones and increasing scatter.

\item None of the SPIRE luminosities are well correlated with the 14--195 keV luminosity, a proxy for the bolometric AGN luminosity. The AGN, in general, is unlikely to be affecting either the 250, 350, or 500 \um emission, however Sy 1s do show a very weak correlation suggesting that perhaps at high AGN luminosity, AGN heated dust could be contributing to the 250 and 350 \um{} emission. The 500 \um{} luminosities for both Sy 1s and Sy 2s are consistent with a null correlation with AGN luminosity.

\item We compared the SPIRE colors, $F_{250}/F_{350}$ and $F_{350}/F_{500}$, with the colors of the HRS galaxies. While the $F_{250}/F_{350}$ colors are statistically different, we found this is due to BAT AGN consisting of only high stellar mass objects. After removing the HRS galaxies with $M_{*} < 9.5 \rm{M_{\odot}}$, the color distributions are statistically similar. The $F_{350}/F_{500}$ color distributions for both samples are statistically similar regardless of stellar mass. This further emphasizes that on average, the FIR emission of AGN host galaxies is likely produced by cold dust in the ISM heated by stellar radiation just as in normal star-forming galaxies without an AGN.

\item We did find anomalous colors for 6 BAT AGN with $F_{250}/F_{350} < 1.5$ and $F_{350}/F_{500} < 1.5$. The FIR SEDs for these AGN are dominated by synchrotron emission from a radio jet rather than thermally heated dust.

\item Another group of AGN with less anomalous colors but still removed from the main locus were analyzed by fitting the SEDs with a modified blackbody and calculating a 500 \um{} excess. We found the 500 \um{} excess is not related to radio loudness, but is well correlated with the 14--195 keV luminosity and $W1/W2$ (3.4/4.6 \um) color from \textit{WISE}. We speculate this is possibly related to the millimeter excess emission recently seen in AGN caused by coronal emission above the accretion disk.
\end{itemize}

Now with all \herschel{} photometry in hand for the BAT AGN, our next step is detailed SED modeling (Shimizu et al, in preparation) to provide a complete picture of the mid-far IR properties of local AGN. We will be able to determine accurate estimates of the dust temperature, dust mass, and SFR for a large, well selected sample of AGN that will reveal answers to key questions. How does the AGN affect star-formation in its host galaxy? What is the AGN contribution to the IR luminosity? How are the IR properties of AGN different from normal galaxies? What is the role of the AGN in galaxy evolution? In this paper, we have hinted at the answers for some of these questions, showing that while AGN host galaxies are quite similar in their long wavelength properties as normal galaxies, a small subset could be affected by the AGN. Full SED modeling however will illuminate our understanding of the IR emission in AGN and our analyses will form the foundation of a local reference sample that can be compared to high-redshift samples in future studies.
  
  
  
  