\subsubsection{Flux Extraction}
We calculate the raw source flux ($F_{raw}$) by summing the values of the pixels within the target aperture. Pixels that are on the border are used by determining the fraction of the pixel area that is inside the aperture and using that fraction of the pixel value in the sum. The background level is determined by estimating the mode of the pixel values within the background annulus using a Python version of the ``MMM'' routine which is the method used in the popular photometry package \texttt{DAOPHOT}. The mode ($F_{bkg-an}$) then represents the per-pixel background level so we multiply it by the pixel area of the target aperture ($A_{src}$) to calculate the total background flux within the target aperture. The background flux is then subtracted from the raw source flux. The whole procedure can be represented with the following equation:
\begin{equation}
F_{\rm{bkg-sub}} = F_{\rm{raw}} - F_{bkg-an} \times A_{src}
\end{equation}

For extended sources, $F_{\rm{bkg-sub}}$ represents the final measured flux density. However for sources which used the point source aperture, we applied the necessary aperture corrections as given in the SPIRE DRG. For the 250, 350, and 500 \um{} bands, these corrections are 1.2697, 1.2271, and 1.2194 respectively. 

Both \citet{Ciesla:2012lq} and \citet{Dale:2012dq} found aperture corrections for extended emission to be small and unnecessary.  To confirm this, we convolved our PACS 160 \um{} images (PSF FWHM 12") to the 250, 350, and 500 \um{} angular resolution using the convolution kernels from \citet{Aniano:2011rr}. This makes the assumption that the 160 \um{} emission is generated by the same material as that producing the SPIRE emission. Aperture corrections were calculated by dividing the total flux within an aperture from the original PACS image by the flux within the exact same aperture applied to the convolved image. The same aperture sizes were used in this calculation as the ones used in this SPIRE analysis. Median aperture corrections of 1.01, 0.98, and 0.98 were found consistent with a value of 1 and confirming that extended emission aperture corrections are not necessary.