

The flux densities are then calculated using the following equation:
\begin{equation}
F_{\rm{bkg-sub}} = F_{\rm{src}} - \frac{F_{bkg-an}}{N_{bkg-an}}N_{src}
\end{equation}
where $F_{\rm{src}}$ and $F_{bkg-an}$ are the total flux densities contained in the source aperture and background annulus and $N_{src}$ and $N_{bkg-an}$ are the total number of pixels in each.

Because the apertures are only measuring the flux within a fraction of the total beam for point sources, aperture corrections must be applied when using aperture photometry for extracting fluxes. While all of the sources we used aperture photometry for were not considered point sources based on their size from timeline fitting procedure, many still fit within the recommended point source aperture of 22", 30", and 42". If an aperture of this size was used for the respective waveband, then a point source aperture correction was applied. For the 250, 350, and 500 \um{} bands, these corrections are 1.295, 1.253, and 1.275 respectively. 

Both \citet{Ciesla:2012lq} and \citet{Dale:2012dq} found aperture corrections for extended emission to be small and unnecessary.  To confirm this, we convolved our PACS 160 \um{} images (PSF FWHM 12") to the 250, 350, and 500 \um{} angular resolution using the convolution kernels from \citet{Aniano:2011rr}. This makes the assumption that the 160 \um{} emission is generated by the same material as that producing the SPIRE emission. Aperture corrections were calculated by dividing the total flux within an aperture from the original PACS image by the flux within the exact same aperture applied to the convolved image. The same aperture sizes were used in this calculation as the ones used in this SPIRE analysis. Median aperture corrections of 1.01, 0.98, and 0.98 were found consistent with a value of 1 and confirming that extended emission aperture corrections are not necessary.
  