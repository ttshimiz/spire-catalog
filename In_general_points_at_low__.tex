In general, points at low values of the $F_{350}/F_{500}$ color show high values of $E_{500}$ and vice versa for high values of the $F_{350}/F_{500}$ color. Points along the main locus are scattered around $E_{500} = 0$. Thus, $E_{500}$ can quantify a source's distance from the main locus and allows us to study possible causes for this excess emission at 500 \um.

We first measure the correlation between $E_{500}$ and radio loudness. AGN historically have been classified into two groups based on how bright their radio emission is compared to another waveband, usually optical. These groups are ``radio-loud'' and ``radio-quiet'' AGN with the former group showing bright radio emission and the latter faint radio emission \citep{Kellermann:1989sf,Xu:1999ty}. While originally radio-loud and radio-quiet AGN seemed to form a dichotomy, the consensus now seems to be that there is a broad distribution of radio-loudness rather than a bimodality \citep{Laor:2003yg,White:2000rz,Cirasuolo:2003rm,Cirasuolo:2003zl,Laor:2003yg}. Further, the original radio loudness parameter, $R = L_{\rm radio}/L_{\rm opt}$ which measured the ratio of the radio to optical luminosity, was shown to underestimate the radio loudness especially for low-luminosity Seyfert galaxies \citep{Terashima:2003fv}. Rather $R_{\rm X} = L_{\rm radio}/L_{\rm X}$ which measures the nuclear radio to X-ray luminosity ratio was confirmed to be a better radio-loudness indicator given X-rays are less affected by obscuration and contamination from the host galaxy. Therefore, for the BAT AGN we use $R_{\rm X}$ to measure the radio-loudness with $L_{\rm radio} = L_{\rm 1.4\,GHz}$ and $L_{\rm X} = L_{\rm 14--195\,keV}$. 

For $L_{\rm 1.4\,GHz}$ we first cross-correlated the BAT AGN with the FIRST and NVSS databases which provide 1.4 GHz flux densities over all of the northern sky. FIRST flux densities were preferred over NVSS due to the much better angular resolution (5" vs. 45"). Since \swift/BAT was an all-sky survey, nearly half of the BAT AGN were not included in either FIRST or NVSS. For these southern sources we turned to the Sydney University Molonglo Sky Survey \citep[SUMSS;][]{Bock:1999fp} which surveyed the southern sky at 843 MHz. Finally, for the remaining sources missing radio data, we performed a literature search and found 5 GHz fluxes from various other studies \citep{Becker:1991qd,Griffith:1993qr,Rush:1996db,Ho:2001hl,Shi:2005rc}. To convert all flux densities to 1.4 GHz, we assumed a power-law spectrum, $F_{\nu} \propto \nu^{-0.7}$, that is typical for synchrotron emission. 
  
  