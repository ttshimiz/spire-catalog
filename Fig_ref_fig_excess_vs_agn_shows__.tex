Fig.~\ref{fig:excess_vs_agn} shows the relationships between both $L_{\rm 14-195\,keV}$ (left panel) and $W1/W2$ (right panel) with $E_{\rm 500}$ after removing the six radio-loud AGN. Both parameters display noticeable correlations with $E_{\rm 500}$ with $L_{\rm 14-195\,keV}$ positively correlated and $W1/W2$ negatively correlated. We calculated Spearman rank correlation coefficients finding values of 0.49 and -0.49 respectively. Pearson correlation coefficients are 0.30 and -0.50 respectively reflecting the more linear relationship between $E_{\rm 500}$ and $W1/W2$ than the one between $E_{\rm 500}$ and $L_{\rm 14-195\,keV}$. In the right panel we also plot the \citet{Stern:2012mz} cutoff for AGN-dominated galaxies where values to the left of this line indicate AGN-dominated colors. 

Both panels indicate that the strength of the AGN in the host galaxy is possibly having an effect on the SPIRE colors. A stronger AGN in relation to the host galaxy is causing deviations from a standard modified blackbody in the form of a small but noticeable 500 \um{} offset.


  
  