Fig.~\ref{fig:excess_vs_agn} shows the relationships between both $L_{\rm 14-195\,keV}$ (left panel) and $W1/W2$ (right panel) with $E_{\rm 500}$ after removing the six radio-loud AGN. Both parameters display noticeable correlations with $E_{\rm 500}$ with $L_{\rm 14-195\,keV}$ positively correlated and $W1/W2$ negatively correlated. We calculated Spearman rank correlation coefficients finding values of 0.49 and -0.49 respectively. Pearson correlation coefficients are 0.30 and -0.50 respectively reflecting the more linear relationship between $E_{\rm 500}$ and $W1/W2$ than the one between $E_{\rm 500}$ and $L_{\rm 14-195\,keV}$. In the right panel we also plot the \citet{Stern:2012mz} cutoff for AGN-dominated galaxies where values to the left of this line indicate AGN-dominated colors. 

Both panels indicate that the strength of the AGN in the host galaxy is possibly having an effect on the SPIRE colors. A stronger AGN in relation to the host galaxy is causing deviations from a standard modified blackbody in the form of a small but noticeable 500 \um{} offset.

Without longer wavelength data, however, its impossible to determine the exact cause of the 500 \um{} excess so we can only speculate. Submillimeter excess emission has been observed in a number of objects including dwarf and normal star-forming galaxies \citep[e.g.][]{Galametz:2009cl,Galametz:2011ao,Dale:2012dq,Remy-Ruyer:2013kx} as well as the Small and Large Magellanic Clouds \citep{Bot:2010zm,Gordon:2010ix} and even our own Milky Way \citep{Paradis:2012oj}. Various explanations have been proposed including the presence of a very cold ($T \sim 10$ K) component \citep{Galametz:2009cl,Galametz:2011ao,OHalloran:2010wt}, grain coagulation that causes the emissivity to increase for colder temperatures \citep{Paradis:2009hb}, fluctuations in the Cosmic Microwave Background \citep{Planck-Collaboration:2011uk}, and an increase in magnetic material in the ISM \citep{Draine:2012vf}. While all of these explanations are certainly still possible to explain the excess seen in the BAT AGN, they lack any direct connection to the strength of the AGN. Further, a key result from all of the previous work is that the submillimeter excess is more prevalent in very metal-poor galaxies ($12 + \log(\rm{O/H} \lesssim 8.3$). All of the BAT AGN reside in high stellar mass galaxies \citep{Koss:2011vn} and given the mass-metallicity relationship \cite{Tremonti:2004fq} should also be quite metal rich. Recent work has also shown that the submillimeter excess can disappear upon reanalysis, especially for more metal rich galaxies \citep{Kirkpatrick:2013wq}.

Rather, we speculate the excess is related to radio emission more closely associated with the AGN itself. Several studies of the radio properties of AGN have revealed a millimeter excess around 100 GHz \citep{Doi:2005wj,Doi:2011si,Behar:2015le,Scharwachter:2015ez} that is likely due to either an inverted or flat SED between cm and mm wavelengths. Because \citet{Doi:2011si} found the excess mainly in low luminosity AGN similar to Sgr A*, they invoked advection dominated accretion flows (ADAF) that produce compact nuclear jets to explain the inverter or flat SEDs. However the sample of \citet{Behar:2015le} was composed of X-ray bright AGN including high Eddington ratio ($L_{\rm bol}/L_{\rm Edd}$), a measure of the accretion rate relative to the Eddington limit) sources where an ADAF is unlikely. \citet{Behar:2015le} instead use the radio-to-X-ray luminosity ratio to argue that the high-frequency radio emission originates near the X-ray corona of the accretion disk given the ratio's similarity to that found for stellar coronal mass ejections \citep[e.g][]{Bastian:1998tx} as well as the compact nature of the radio emission. Magnetic activity around the accretion disk in the core of the AGN would then be responsible for the excess and if magnetic activity increases with $L_{\rm bol}/L_{\rm Edd}$, this could explain the relationship seen with $L_{\rm 14-195\,keV}$ as well as $W1/W2$. This strengthens the need for a more comprehensive survey of AGN in the mm wavelength range as it could clearly reveal interesting physics possibly occurring near the accretion disk.
  
  
  
  
  