While all the relationships show some amount of correlation, the strongest ones occur between wavelengths that are nearest each other. The 160 vs. 250 \um{} and 250 vs. 350 \um{} correlations have a correlation coefficient of $\sim0.7$. This makes sense within the context of multiple temperature components. Photometry from nearby wavelengths should be produced from closely related temperature components.

The weak correlation between 70 and 500 \um{} indicates the emission in these wavebands does not originate from closely related processes. 70 \um{} emission comes from much hotter dust than 500 \um{} and several processes could provide an explanation. Since this is an AGN sample, there could be a strong contribution from AGN heated dust, whereas at 500 \um{}, AGN related emission would be negligible. This is supported by our findings in \citet{Melendez:2014yu} where we showed that 70 \um{} emission of the BAT AGN was weakly correlated with AGN luminosity. Further, in \citet{Mushotzky:2014ad} we found that the BAT AGN morphologies at 70 \um{} were concentrated in the nucleus potentially indicating an AGN contribution.

The weak correlation, however, can also be explained if non-star-forming processes also contribute to the 500 \um{} emission. While, in non-AGN galaxies, the majority of 70 \um{} emission is most likely heated by UV radiation from recently formed massive stars, 500 \um{} emission can be produced by heating from relatively old stars. Therefore, the disconnect between the stellar populations would produce significant scatter in the correlation between 70 and 500 \um. Also, synchrotron radiation produced by radio jets associated with AGN can contribute to the FIR, especially the longest wavelengths as seen in some radio-loud galaxies \citep{Baes:2010ek,Boselli:2010fr}. This non-thermal emission would be completely unrelated to the thermal emission at 70 \um, thereby producing a weaker correlation between the luminosities at those wavebands. In a later section we will show there are indeed some radio-loud sources in our sample where synchrotron emission dominates the SPIRE emission. 

When we break the sample down into Sy 1's and 2's we don't find much difference between the correlation coefficients. This shows that Sy 1's and 2's are not different in terms of their overall FIR emission and the same processes are likely producing the FIR emission. Sy 1's do show a slightly weaker correlation between the \textit{Herschel} luminosities especially the ones involving 500 \um. This is likely due to the fact that most radio-loud AGN are classified as Sy 1's so synchrotron emission is contributing strongest at 500 \um{} compared to the other wavebands.
  