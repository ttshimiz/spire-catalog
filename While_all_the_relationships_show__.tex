While all the relationships show some amount of correlation, the strongest ones occur between wavelengths that are nearest each other. The 160 vs. 250 \um{} and 250 vs. 350 \um{} correlations have a correlation coefficient of $\sim0.7$. This makes sense within the context of multiple temperature components. Photometry from nearby wavelengths should be produced from closely related temperature components.

The weak correlation between 70 and 500 \micron{} indicates the emission in these wavebands does not originate from closely related processes. 70 \micron{} emission comes from much hotter dust than 500 \micron{} and several processes could provide an explanation. Since this is an AGN sample, there could be a strong contribution from AGN heated dust, whereas at 500 \micron{}, AGN related emission would be negligible. This is supported by our findings in \citet{Melendez:2014yu} where we showed that 70 \micron{} emission of the BAT AGN was weakly correlated with AGN luminosity. Further, in \citep{Mushotzky:2014ad} we found that the BAT AGN morphologies at 70 \micron{} were concentrated in the nucleus potentially indicating an AGN contribution.

The weak correlation, however, can also be explained given 500 \micron{} emission is not necessarily related to star formation. While, in non-AGN galaxies, the majority of 70 \micron{} emission is most likely heated by UV radiation from recently formed massive stars, 500 \micron{} emission can be produced by heating from relatively old stars. Therefore, the disconnect between the stellar populations would produce significant scatter in the correlation between 70 and 500 \micron.