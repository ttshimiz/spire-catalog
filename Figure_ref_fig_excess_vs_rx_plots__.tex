Figure~\ref{fig:excess_vs_rx} plots $E_{\rm 500}$ against $R_{\rm X}$ to test our theory that the excess 500 \um{} emission is related to the radio loudness of the AGN. In the left panel we plot all of the sources together to show the full range of $E_{\rm 500}$. Indeed, the six AGN with the largest values of $E_{\rm 500}$ exhibit high values of radio loudness ($\log\,R_{\rm X} > -4.0$). These six AGN are HB 890241+622, 2MASX J23272195+1524375, 3C 111.0, 3C 120, Pictor A, and PKS 2331-240 and all are well known radio-loud AGN and correspond the six sources in Figures~\ref{fig:color-color} and \ref{fig:color-color_excess} that lie in the lower left hand corner. Further, the lone HRS galaxy seen in Figures~\ref{fig:color-color} among the six BAT AGN is the radio galaxy M87, whose jets and radio activity have been studied extensively. Based on this, we prescribe color cutoffs that can easily separate radio-loud AGN from radio-quiet AGN and normal star-forming galaxies: $F_{250}/F_{350} < 1.5$ and $F_{350}/F_{500} < 1.5$. 

While radio-loudness can explain the most extreme values of $E_{\rm 500}$, it does not explain the more moderate ones. In the right panel of Figure~\ref{fig:excess_vs_rx}, we zoom in on the AGN with $E_{\rm 500} < 1.0$. Visually there does not appear to be any strong correlation between $R_{\rm X}$ and $E_{\rm 500}$ and the Spearman rank correlation coefficient between them is -0.15, weak and in the opposite sense of what would be expected if synchrotron emission was contaminating the 500 \um{} emission.
  
To explore even further, we analyzed the correlations between $E_{\rm 500}$ and two AGN-related indicators, the \swift/BAT luminosity, $L_{14-195\,keV}$, and the 3.4 to 4.6 \um{} flux ratio ($W1/W2$). The 3.4 ($W1$) and 4.6 ($W2$) \um{} fluxes for the BAT AGN were obtained from the \textit{Wide-field Infrared Survey Explorer} \citep[\textit{WISE};][]{Wright:2010fk} AllWISE catalog accessed through the \textit{NASA/IPAC} Infrared Science Archive (IRSA)\footnote{\url{http://irsa.ipac.caltech.edu/Missions/wise.html}}. Details of the compilation of \textit{WISE} fluxes for the BAT AGN will be available in an upcoming publication (Shimizu et al. in preparation).

\citet{Winter:2012yq} showed that $L_{14-195\,keV}$ can be used as a measure of the intrinsic bolometric luminosity of the AGN, unaffected by host galaxy contamination or line-of-sight absorption. $W1/W2$ has been shown to be an effective discriminator between AGN-dominated and normal star-forming galaxies that has both high reliability and completeness \citep{Stern:2012mz}\footnote{\citet{Stern:2012mz} prescribe a cutoff of $W1 - W2 \geq 0.8$  in magnitude units for selecting AGN. In flux units this changes to $W1/W2 \leq 0.86$}. \citet{Stern:2012mz} also show that as the fraction of emission coming from the host galaxy increases $W1/W2$ increases as well making it a good measure of the relative contribution of the AGN to the infrared luminosity. 
  