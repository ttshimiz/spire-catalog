Figure~\ref{fig:excess_vs_rx} plots $E_{\rm 500}$ against $R_{\rm X}$ to test our theory that the excess 500 \um{} emission is related to the radio loudness of the AGN. In the left panel we plot all of the sources together to show the full range of $E_{\rm 500}$. Indeed, the six AGN with the largest values of $E_{\rm 500}$ exhibit high values of radio loudness ($\log\,R_{\rm X} > -4.0$). These six AGN are HB 890241+622, 2MASX J23272195+1524375, 3C 111.0, 3C 120, Pictor A, and PKS 2331-240 and all are well known radio-loud AGN and correspond the six sources in Figures~\ref{fig:color-color} and \ref{fig:color-color_excess} that lie in the lower left hand corner. Based on this, we prescribe color cutoffs that can easily separate radio-loud AGN from radio-quiet AGN and normal star-forming galaxies: $F_{250}/F_{350} < 1.5$ and $F_{350}/F_{500} < 1.5$. 

While radio-loudness can explain the most extreme values of $E_{\rm 500}$, it does not explain the more moderate ones. In the right panel of Figure~\ref{fig:excess_vs_rx}, we zoom in on the AGN with $E_{\rm 500} < 1.0$. Visually there does not appear to be any strong correlation between $R_{\rm X}$ and $E_{\rm 500}$ and the Spearman rank correlation coefficient between them is -0.15, weak and in the opposite sense of what would be expected if synchrotron emission was contaminating the 500 \um{} emission.
  
  