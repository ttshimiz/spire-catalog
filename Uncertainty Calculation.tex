\subsubsection{Uncertainty Calculation}
Three components were factored into the total error budget for the SPIRE photometry of our sample. These were the instrumental error ($err_{\rm{inst}}$), background error ($err_{\rm{bkg}}$), and calibration error ($err_{\rm{cal}}$). $err_{\rm{cal}}$ is fixed at 9.5\% of the measured background-subtracted flux density \citep{Bendo:2013sd}. To determine $err_{\rm{inst}}$, we summed in quadrature all of the 1$\sigma$ pixel uncertainties from the error map contained in the source aperture. For $err_{\rm{bkg}}$, we placed 6 equally spaced apertures around the source aperture with the same size as the source aperture. The root mean square of the fluxes from these 6 apertures was calculated and used as an estimate for $err_{\rm{bkg}}$. For the sources with elliptical apertures, more than 6 apertures were used with a size ranging from 22"--120" and placed to fill up the region around the source. Again, the root mean square of the fluxes from these apertures were calculated, but then scaled to the number of pixels in the source aperture by multiplying by $N_{\rm{src}}/{N_{\rm{bkg-ap}}}$ where $N_{\rm{bkg-ap}}$ is the average number of pixels in the background apertures. The three error components are then summed in quadrature to form the total 1$\sigma$ uncertainty ($err_{\rm{tot}}$) of the measured flux density for each source.