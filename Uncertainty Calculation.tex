\subsection{Uncertainty Calculation}
For sources where we used aperture photometry, three components were factored into the total error budget for the SPIRE aperture photometry of our sample. These were the instrumental error ($err_{\rm{inst}}$), background error ($err_{\rm{bkg}}$), and calibration error ($err_{\rm{cal}}$). $err_{\rm{cal}}$ is fixed at 6.5\% of the measured background-subtracted flux density for sources which used aperture photometry. The calibration error is the combination of the 4\% uncertainty in the Neptune flux model, the 1.5\% uncertainty from repeated measurements of Neptune, and the 1\% uncertainty in the beam areas \citep{Bendo:2013sd}. To determine $err_{\rm{inst}}$, we summed in quadrature all of the 1$\sigma$ pixel uncertainties from the error map contained in the target aperture. For $err_{\rm{bkg}}$, we measured the flux within the circular background apertures placed around the source aperture. The standard deviation of the fluxes was calculated after using sigma-clipping with a 3$\sigma$ cutoff to remove fluxes possibly contaminated with a bright, background source.  This was then scaled to the area of the target aperture to represent $err_{\rm{bkg}}$.  The three error components are then summed in quadrature to form the total 1$\sigma$ uncertainty ($err_{\rm{tot}}$) of the measured flux density for each source.

For sources where we used the timeline fitting, only two components are needed. The output from the timeline fitting contains an estimate of the statistical uncertainty in the flux density. This is combined in quadrature with a 5.5\% calibration error, which is the same as the calibration error for aperture photometry minus the 1\% uncertainty in the beam areas that are not needed in the timeline fitting.
  