\section{The \swift/BAT AGN Sample}\label{sample}
The  \swift/\textit{Burst Alert Telescope} (BAT) \cite{Barthelmy_2005,Gehrels_2004} operates in the 14--195 keV energy range, continuously monitoring the sky for gamma-ray bursts. This constant monitoring has also allowed for the most complete \href{https://swift.gsfc.nasa.gov/results/bs70mon/}{all-sky survey} in the ultra-hard X-rays. To date, BAT has detected 1171 sources at  $>4.8\sigma$ significance corresponding to a sensitivity of $1.34\times10^{-11}$ ergs s$^{-1}$ cm$^{-2}$. Over 700 of those sources have been identified as a type of AGN (Seyfert, Blazar, QSO, etc.)

We selected our sample of 313 AGN from the 58 month \swift/BAT Catalog \citep{Baumgartner:2012gf}, imposing a redshift cutoff of $z<0.05$. All different types of AGN were chosen only excluding Blazars/BL Lac objects which most likely introduce complicated beaming effects. In total the sample contains 94 Sy 1, 20 Sy 1.2, 1 Sy 1.4, 34 Sy 1.5, 9 Sy 1.8, 17 Sy 1.9, 129 Sy 2, 5 LINERs, and 4 AGN. For the purpose of broad classification, in the rest of this paper we choose to classify all Sy 1-1.5 as Seyfert 1's, and all Sy 1.8-2 as Seyfert 2's. In Table \ref{tbl_1} we list the entire \herschel{}-BAT sample along with positions and redshifts taken from the \href{http://ned.ipac.caltech.edu/}{\textit{NASA/IPAC Extragalactic Database} (NED)}. Luminosity distances for each AGN were calculated based on their redshift, except for those with $z < 0.01$ where we referred to the \href{http://edd.ifa.hawaii.edu/]}{\textit{Extragalactic Distance Database}}.

Selection of AGN by ultra-hard X-rays provides multiple advantages over other wavelengths. Due to their high energy, ultra-hard X-rays easily pass through any gas or dust in the line of sight providing a direct view of the AGN. Using optical or mid-infrared selection can be problematic due to contamination by the host galaxy. Also ultra-hard X-rays are unaffected by any type of absorption by material obscuring the AGN provided it is optically thin to Compton scattering ($N_{\rm{H}} \lesssim 10^{24}\,\, \rm{cm}^{-2}$) which is a concern for hard X-rays in the 2-10 keV energy range.