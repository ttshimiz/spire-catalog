\subsection{FIR Properties of the \herschel-BAT Sample }
\subsubsection{Detection Rate and Luminosity Distributions}\label{sec:det_rate_lum_dist}
\citet{Melendez:2014yu} in analyzing the PACS photometry found 95\% and 83\% of the BAT sample had a 5$\sigma$ detection at 70 and 160 \um, indicating a largely complete survey of AGN for those wavelengths. Our SPIRE analysis finds a 5$\sigma$ completeness of 86\%, 72\%, and 46\% for 250, 350, and 500 \um{} respectively. The decreasing completeness reflects both the decreasing sensitivity of SPIRE with increasing wavelength as well as the rapid fall-off of the SED at longer wavelengths. Even with the relatively low detection rate at 500 \um, this still results in 143 AGN having complete FIR SEDs from 70--500 \um, representing a great step forward in advancing the study of the dust content in AGN. 

After splitting the sample into Sy 1's and Sy 2's, we find a distinct difference in the detection rate (Figure~\ref{fig:det_frac}). Sy 2's, for all 3 wavebands, are detected at a significantly higher rate than Sy 1's (95\% vs. 80\% for 250 \um, 85\% vs. 62\% for 350 \um, 59\% vs. 34\% for 500 \um).