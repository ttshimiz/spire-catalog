\subsubsection{Wavelength--Wavelength Luminosity Correlations}
Dust at multiple temperatures is thought to produce the FIR SED \citep{Draine:2003gd}. Shorter wavelength emission corresponds to hotter components. Both the amount of dust (i.e. dust mass) heated to a specific temperature as well as the relative intensity of the heating process determine the strength of the emission at a particular wavelength. If the same process is heating all of the dust and producing the entire FIR SED, we would expect strong correlations between each wavelength. By studying the correlations between the various \textit{Herschel} luminosities we can begin to determine whether the same process is heating the dust at various temperatures.

Three processes could contribute to the heating of dust in the BAT AGN. Recent star formation in the galaxy will produce OB stars with a high intensity of UV light that can heat nearby dust to large temperatures. UV light can also escape the star-forming regions and heat dust further away to colder temperatures. Older stellar populations, however, also produce an interstellar radiation field that can heat diffuse dust to temperatures around 15 K which would contribute most heavily at the longest wavelengths. Finally, the UV light from the AGN itself can possibly heat dust in the circumnuclear region.

We ran a correlation analysis between each \textit{Herschel} waveband. Two effects must be taken into account to establish reliable correlation coefficients: censoring and confounding variables. The confounding variable in this case is distance. Since our sample is flux-limited, higher luminosity objects are more likely to be found at larger distances. Therefore it can produce the effect of an intrinsic correlation when comparing two luminosities. To mitigate the effects of censoring and the luminosity-distance relationship, we calculated the partial Kendall-$\tau$ correlation coefficient as presented in \citet{Akritas_1996}. Table~\ref{tab:wave_corrs} displays all of the correlation coefficients ($\rho_{\tau}$) as well as the probability of zero correlation ($P_{\tau}$). 