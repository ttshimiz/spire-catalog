\section{Introduction}\label{sec:intro}
The star formation rate (SFR) of galaxies sets the rate at which galaxies grow and evolve and is the one of the most important measures for understanding the hierarchical build-up of our universe over cosmic time. Large scale simulations, however, have shown that unregulated star formation leads to an overabundance of high mass galaxies \citep[e.g.][]{Bower:2006gf,Croton:2006kx,Silk:2012fj}. Therefore some process (or processes) must be able to stop, or ``quench,'' star formation before the galaxy grows to be too big.

The answer seems to lie in supermassive black holes (SMBH) which nearly all massive galaxies harbor in their centers. SMBHs grow through accretion of cold material (Active Galactic Nuclei; AGN), and the huge loss of gravitational energy of the cold material is converted into radiation that is evident across the whole electromagnetic spectrum and manifests itself as a bright point source in the nucleus of galaxies. The AGN can deposit this energy into the ISM of its host galaxy through jets \cite[e.g.][]{Fabian:2003ek,Best:2007vn,Lanz:2015bq} or powerful outflows \citep[e.g][]{Alatalo:2011lk,Veilleux:2013qq,Harrison:2014xe,Tombesi:2015fj} that either heat the gas or remove it altogether, i.e. ``feedback.''
  
Indirect evidence of this ``feedback'' has been observed through the simple, scaling relationships between the mass of the SMBH and different properties of the host galaxy such as the stellar velocity dispersion in the bulge, the bulge mass, and the bulge luminosity \citep[e.g.][]{Kormendy:1995mz,Ferrarese:2000gf,Marconi:2003ve,Haring:2004ly,Gultekin:2009ul,Kormendy:2013fj}. The relative tightness of these relationships suggests a strong coevolution of the host galaxy and SMBH. Much debate remains however as to the exact mechanism of AGN feedback and whether or not it plays a dominant role in the overall evolution of galaxies especially in light of new observations at both low and high $M_{\rm BH}$ that seem to deviate from the well-established relationships \citep[see][for a detailed review]{Kormendy:2013fj}. 

Evidence for AGN feedback though should also manifest itself in the SFR of its host galaxy, therefore much work has also focused on the so-called starburst-AGN connection \citep[e.g.][]{Sanders:1988fk,Cid-Fernandes:2001uq,Diamond-Stanic:2012rw,Dixon:2011yq,Rovilos:2012wd,Chen:2013uq,LaMassa:2013hb,Esquej:2014vl,Hickox:2014yq,Mushotzky:2014ad}. The problem lies in determining accurate estimates of the SFR in AGN host galaxies. Well calibrated indicators, such as H$\alpha$ emission and UV luminosity, are significantly, if not completely, contaminated by the central AGN. Many studies therefore turn to the infrared (IR) regime ($1<\lambda<1000$ \um) where dust re-emits the stellar light from young stars.

Dust fills the interstellar medium (ISM) of galaxies and plays an important part in the heating and cooling of the ISM and the general physics of the galaxy. While dust contributes very little to the overall mass of a galaxy ($<1\%$), the radiative output, mainly in the infrared (IR) regime, can constitute roughly half of the bolometric luminosity of the entire galaxy \citep{Hauser_2001,Boselli_2003,Dale:2007fk,Burgarella_2013}. Dust efficiently absorbs optical and UV emission and re-radiates it in the mid and far-infrared (MIR, FIR) depending on the temperature as well as grain size \citep{Draine:2003gd}. Recently formed O and B stars produce the majority of the optical and UV light in galaxies, therefore measuring the total IR light from dust provides insights into the current ($<100$ Myr) star formation rate (SFR) \citep[e.g.][]{Kennicutt:2012it}.

However, dust is also the key component in obscuring our view of AGN. Dust is thought to live in a toroidal-like structure that encircles the AGN and absorbs its radiative output for certain lines of sight. The dusty torus is used to explain the dichotomy of AGN into Seyfert 1 (Sy 1) and Seyfert 2 (Sy 2) within a unified model \citep{Antonucci:1993os,Urry:1995il}. Like O and B stars in star-forming regions, the AGN outputs heavy amounts of optical and UV light, and like dust in the ISM the dusty torus absorbs and re-emits this as IR radiation. Spectral energy distribution (SED) models \citep{Barvainis:1987ty,Pier:1992sf,Efstathiou:1995rz,Nenkova:2002ys,Fritz:2006yq}  as well as observations \citep{Elvis:1994uq,Spinoglio:2002uq,Netzer:2007ve,Mullaney:2011yq,Mor:2012fj} suggest the torus mainly emits in the MIR ($3<\lambda<40$ \um) with the flux density dropping rapidly in the FIR ($\lambda>40$ \um). Further the SED for stellar dust re-radiation peaks in the FIR \citep{Calzetti:2000fk,Dale:2002ty,Draine:2007rm}, making the FIR the ideal waveband to study star-formation in AGN host galaxies.

Space-based telescopes such as the \textit{Infrared Astronomical Satellite} \cite[IRAS;][]{Neugebauer:1984fp}, \textit{Spitzer Space Telescope} \citep{Werner:2004cr}, and \textit{Infrared Space Observatory} \citep{Kessler:1996wd} greatly expanded our knowledge of the IR universe and provided a window into the FIR properties of galaxies. But, before the launch of the \textit{Herschel Space Observatory} \citep{Pilbratt:2010rz}, the FIR SED was limited to $\lambda < 200$ \um. \herschel{} with the Spectral and Photometric Imaging Receiver \citep[SPIRE;][]{Griffin:2010sf} has pushed into the submillimeter range with observations in the 250, 350, and 500 \um{} wavebands, probing the Rayleigh-Jeans tail of the modified blackbody that accurately describes the broadband FIR SED of galaxies \citep[e.g.][]{Calzetti:2000fk,Dale:2012dq,Cortese:2014qq}. These wavebands are crucial for measuring dust properties (i.e. temperature and mass) as \citet{Galametz:2011ao} and \citet{Gordon:2010ix} show. Further, \citet{Ciesla:2015qr} found that FIR and submillimeter data are crucial for estimating the SFR of AGN host galaxies.

Therefore, we have assembled a large ($\sim300$), low redshift ($z<0.5$) sample of AGN selected using ultra-hard X-ray observations with the \swift/\textit{Burst Alert Telescope} (BAT) and imaged each one with \herschel. In \citet{Melendez:2014yu}, we presented the PACS data of the BAT AGN which provided photometry at 70 and 160 \um. In this paper, we complete the FIR SED of the BAT AGN with the creation and analysis of the SPIRE images. We focus on the overall luminosity distributions at the SPIRE wavebands as well as the SPIRE colors ($F_{250}/F_{350}$ and $F_{350}/F_{500}$) to determine the likely heating sources of cold dust in AGN host galaxies. We also look for correlations with a proxy for the bolometric AGN luminosity to potentially reveal any indication that AGN heated dust is contributing to the FIR SED. This paper sets us up for a complete study of the mid-far IR SED to fully explore the star-forming properties of AGN host galaxies and reveal the global starburst-AGN connection in the nearby universe. Throughout this paper we assume a $\Lambda$CDM cosmology with $H_0=70$ km s$^{-1}$ Mpc$^{2}$, $\Omega_{m} = 0.3$, and $\Omega_{\Lambda}=0.7$.  Luminosity distances for each AGN were calculated based on their redshift and assumed cosmology, except for those with $z < 0.01$ where we referred to the \href{http://edd.ifa.hawaii.edu/]}{\textit{Extragalactic Distance Database}}.
  