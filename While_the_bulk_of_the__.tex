While the bulk of the SPIRE colors are very similar between the HRS and BAT, and the two Seyfert types, one noticeable difference is a distinct bump in the color distribution around 0.75. This bump is absent in the HRS sample and mainly is made up of Sy 1s. With both flux ratios less than one, this indicates a monotonically rising SED that is in stark contrast with the rapidly declining SED characteristic of a modified blackbody. The equation for a modified blackbody is
\begin{equation}\label{eq:mod_blackbody}
F_{\nu} \propto \nu^{\beta}B(\nu, T)
\end{equation}
\noindent where $B(\nu, T)$ is the standard Planck blackbody function with a temperature of $T$. In Figure~\ref{fig:color-color} we plot both colors together for the HRS and BAT AGN. Nearly all of the HRS galaxies are concentrated along a main locus as well as many of the BAT AGN. We also plot the expected colors for a modified blackbody with varying temperature between 10 and 60 K and an emissivity of either 2.0 (green line and squares) or 1.5 (purple line and diamonds). Each square or diamond represents an increase of 5 K starting at 10 K in the lower left. The main locus for both samples is clearly aligned with a modified blackbody with temperatures between 15--30 K. \cite{Cortese:2014qq} fit the FIR SED of the HRS sample using a single temperature modified blackbody finding exactly this range of temperatures and an average emissivity of 1.8. Further these values are consistent with dust in the Milky Way, Andromeda, and other nearby galaxies \citep{Galametz:2012uq, Boselli:2012qv, Smith:2012fj}. 

However the bump seen in Figure~\ref{fig:hist_colors} becomes very evident in Figure~\ref{fig:color-color} as a separate population in the lower left-hand corner. Specifically 6 BAT AGN and one HRS galaxy occupy the region of color-color space where $F_{250}/F_{350} < 1.5$ and $F_{350}/F_{500} < 1.5$. Based on the theoretical curves, these exceptional colors cannot be explained as either a different temperature or emissivity. Rather an entirely different process is producing the FIR emission in these galaxies and since the colors indicate essentially a rising SED, we suspected synchrotron radiation as the likely emission mechanism with its characteristic increasing power law shape with wavelength.
  
  
  