\subsection{Comparison with \textit{Planck}}
We also compared our fluxes with those from the \textit{Planck} Catalog of Compact Sources (PCCS, \citet{Planck-Collaboration:2013rt}. The \textit{Planck} telescope performed an all-sky survey at 9 submilleter and radio wavebands to primarily measure the cosmic microwave background. The highest frequency band centered at 857 GHz matches the SPIRE 350 \um{} waveband and the 545 GHz (550 \um{}) overlaps the SPIRE 500 \um{} waveband allowing for independent measurements of the flux density of our sources. We searched the PCCS for our sources at each frequency using a 4' search radius and found 60 matches at 350 \um{} and 37 at 500 \um{}. To be consistent with our work we chose the aperture fluxes to compare with ours except for Centaurus A, NGC 1365, and M106 in which we chose the fluxes from fitting a Gaussian. These three sources are resolved even with \textit{Planck}'s poor spatial resolution, so the aperture fluxes will underestimate the true flux because the aperture sizes are equal to the resolution at each frequency. 

We applied color corrections to the \textit{Planck} fluxes to account for the differences in both central wavelength and spectral response. These were downloaded from the \href{https://nhscsci.ipac.caltech.edu/sc/index.php/Spire/PhotDataAnalysis}{NHSC website} and provides corrections for different temperature greybodies with an assumed emissivity of 1.8. Also provided are $F_{545}/F_{847}$ flux ratios that correspond to each temperature, which we compare with each observed flux ratio to find the right color correction for each source. Therefore we also restricted our comparison to only include sources that were detected in both the 545 and 857 GHz band giving a total of 27 sources. 

After correcting the \textit{Planck} fluxes, we compare them to our SPIRE fluxes and find a median SPIRE-to-\textit{Planck} ratio of 0.90 and 1.00 for the 350 and 500 \um{} band respectively. This shows a relatively good agreement between the SPIRE and \textit{Planck} instruments, especially in the 500 \um{} band. The $\sim$10\% difference in the 350 \um{} band is most likely due to the large aperture sizes used for \textit{Planck} photometry that will include a much greater amount of diffuse Milky Way cirrus emission. The discrepancy is also still well within the errors for the SPIRE and \textit{Planck} fluxes so we conclude they are consistent with each other. 
  