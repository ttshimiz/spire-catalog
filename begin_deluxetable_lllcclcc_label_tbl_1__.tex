\begin{deluxetable}{lllcclcc}\label{tbl_1}
\tabletypesize{\scriptsize}
\tablecaption{The \herschel-BAT Sample}
\tablehead{\colhead{Name} & \colhead{RA} & \colhead{DEC} & \colhead{$z$} & \colhead{Distance} &\colhead{Type}  & \colhead{OD} & \colhead{OBSID}\\
\colhead{} & \colhead{(J2000)} & \colhead{(J2000)} & \colhead{} & \colhead{(Mpc)} & \colhead{} & \colhead{} & \colhead{}}
\startdata
Mrk 335	&	00h06m19.5s	&	+20d12m10s	&	0.0258	&	112.62	&	Sy 1.2    	&	949	&	     1342234683\\
2MASX J00253292+6821442	&	00h25m32.9s	&	+68d21m44s	&	0.012	&	51.87	&	Sy 2      	&	1022	&	     1342239794\\
CGCG 535-012	&	00h36m21.0s	&	+45d39m54s	&	0.0476	&	211.45	&	Sy 1.2    	&	976	&	     1342237509\\
NGC 235A	&	00h42m52.8s	&	-23d32m28s	&	0.0222	&	96.83	&	Sy 1      	&	737	&	     1342221462
\enddata
\tablecomments{\textit{Column 1:} Name of the source. \textit{Column 2:} Right ascension in J2000 coordinates. \textit{Column 3:} Declination in J2000 coordinates. \textit{Column 4:} Redshift of the source. \textit{Column 5:} Luminosity distance of the source in Mpc. \textit{Column 6:} AGN type from the \swift/BAT 70 month catalog. \textit{Column 7:} \herschel{} Operational Day number for when the observation started. \textit{Column 7:} \herschel{} Observation Identification number. Full table available in the online version}
\end{deluxetable} 
  
  
  