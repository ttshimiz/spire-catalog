\subsection{Comparison with \herschel{} Reference Survey}
The \herschel{} Reference Survey (HRS, \citet{Boselli:2010fj}) is a guaranteed time key project that surveyed 323 nearby ($15<D<25$ Mpc) galaxies using SPIRE to explore the dust content in early and late-type galaxies. Cross-correlating our sample with HRS, we found 4 sources (NGC 3227, NGC 4388, NGC 4941, and NGC 5273) that are a part of both. We compared the fluxes published in \citet{Ciesla:2012lq} to our own and find a mean ratio $F_{\rm{BAT}}/F_{\rm{HRS}}$ of 0.88, 0.89, and 0.88 for the 250, 350, and 500 \um{} wavebands respectively. 

However, there are several distinct differences between the HRS analysis and ours with one major difference being the beam area sizes. \citet{Ciesla:2012lq} used beam areas of 423, 751, and 1587 arcsec$^{2}$ compared with 465, 822, and 1769 arcsec$^{2}$ for our analysis. To correct for this, we multiplied the HRS fluxes for the galaxies by 423/465, 751/822, and 1587/1769 for the 250, 350, and 500 \um{} bands respectively.  After this correction the flux comparison ratios change to 0.97, 0.98, and 0.98. The remaining few percent differences we attribute to the differences in observing mode, map maker (Scanamorphos vs. naive map), data reduction, and photometry techniques.