\subsection{Timeline Fitting}\label{timeline}
Timeline fitting involves modeling the response of a point source in the Level 1 data as a two dimensional Gaussian and determining the best fit parameters for the Gaussian. The peak of the Gaussian then corresponds to the flux density of the source. Because this method is performed on the Level 1 data, instead of the image maps, it avoids any potential artifacts or biases involved with the mapmaking procedure and is the highly recommended procedure for determining the photometry of point sources by the \href{http://herschel.esac.esa.int/hcss-doc-11.0/index.jsp\#spire_drg:_start}{SPIRE Data Reduction Guide} (DRG, section 5.7.1).

To determine which sources are unresolved, we first visually separated the entire sample into point-like and extended sources. Each source was also classified as ``Detected'' or ``Undetected'' at each waveband. Then all of the point-like sources' Level 1 data were fit using the \texttt{sourceExtractorTimeline} task within HIPE to measure the best fit Gaussian where one of the free parameters is the size of the source, represented by the FWHM of the Gaussian. A source is then considered unresolved in a waveband if its best fit FWHM is less than 19'', 25", or 36" at 250, 350, or 500 \micron{} respectively, or $\sim$1" bigger than the FWHM reported in the SPIRE DRG. However to avoid combining different flux extraction techniques for a single source, we only used the timeline flux densities if that source was unresolved at all wavebands in which it is visually detected.