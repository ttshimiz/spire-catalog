\subsection{Timeline Fitting}\label{timeline}
Timeline fitting involves modeling the response of a point source in the Level 1 data as a two dimensional Gaussian and determining the best fit parameters for the Gaussian. The peak of the Gaussian then corresponds to the flux density of the source. Because this method is performed on the Level 1 data, instead of the image maps, it avoids any potential artifacts or biases involved with the mapmaking procedure and is the highly recommended procedure for determining the photometry of point sources by the \href{http://herschel.esac.esa.int/hcss-doc-11.0/index.jsp\#spire_drg:_start}{SPIRE Data Reduction Guide} (DRG, section 5.7.1).

To determine which sources are unresolved, we fit the Level 1 data using the \texttt{sourceExtractorTimeline} task within HIPE to measure the best-fit Gaussian where one of the free parameters is the size of the source, represented by the FWHM of the Gaussian. A source is then considered unresolved in a waveband if its best fit FWHM is less than 21'', 28", or 40" at 250, 350, or 500 \um{} respectively, the upper limit for the nominal ranges of FWHM expected for point sources. We also employ a maximum reduced $\chi^{2}$ cutoff of 0.0022, 0.0023, and 0.0034 to avoid extended sources that have bright central point sources. The cutoff values were determined from analyzing the distribution of reduced $\chi^{2}$ for all of the BAT sources as well as visually inspecting the images to make sure they are point sources. To avoid combining different flux extraction techniques for a single source, we only used the timeline flux densities if that source was unresolved at all wavebands in which it is visually detected.

We used the timeline fluxes for 85, 65, and 41 sources in each of the three wavebands. These sources are indicated in Table 2 with a ``TF". The discrepancy comes from some of the sources being undetected at longer wavelengths due to the rapid fall-off of the SED as well as a decreasing sensitivity. For the sources that are undetected we use aperture photometry to determine their $5\sigma$ upper limit. 

  
  