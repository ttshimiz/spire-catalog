\section{Results}
Table 2 represents our final SPIRE catalog for the \swift/BAT AGN. For each waveband 3 columns are provided. The first contains the flux density and $1\sigma$ uncertainty. The second column provides the photometry method used to determine the flux density, either timeline fitting ('TF') or aperture photometry ('AP'). The third column provides flags to assist in assessing the reliability of the photometry. We decided to impose a strict 5$\sigma$ threshold for reporting the photometry, so for all sources where $5 err_{\rm{tot}} > F_{\rm{bkg-sub}}$, only the 5$\sigma$ upper limit is given as the flux density for that band and a flag of 'U' is used. For sources above $5\sigma$ a flag of 'A' is used. Alongside these two flags we also indicate those sources that are contaminated by foreground cirrus emission with a flag of 'C'. Finally a flag of 'd' or 'D' is used for sources that have a nearby companion that could possibly be affecting the photometry of the main BAT source. 'd' represents companions that are either relatively faint compared to the BAT source or are far enough away where contamination to the BAT SPIRE photometry is minimal. 'D' represents nearby bright companions that are completely contaminating the source photometry and recommend using these flux densities as only upper limits. In total 17 and 10 sources have a 'd' and 'D' classification for the 250 \um{} waveband, 20 and 13 for the 350 \um{} waveband, and 13 and 21 for the 500 \um{} waveband. The changing numbers with wavelength represents the degrading resolution as wavelength increases.