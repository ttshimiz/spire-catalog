\subsubsection{Source Detection}
The first step in the process is converting the SPIRE map units from Jy/beam to Jy/pixel. The images must be divided by the beam area specific to the waveband and calibration version (spire\_cal_13\_1) used to create the maps. For this work the beam areas are 469.7, 831.7, and 1793.5 arcsec$^{2}$ for the 250, 350, and 500 \um{} wavebands respectively and taken from the latest version of the SPIRE DRG. Each pixel of the map is converted to Jy/pixel using the following formula.

\begin{equation}
I\,[Jy/pixel] = \frac{I\,[Jy/beam] \times P^2}{B}
\end{equation}

\noindent $I$ is the intensity value of the pixel, $P$ is the pixel size of the map (see Section~\ref{sec:obs}), and $B$ is the beam area stated above. After converting all of the pixels, we use an iterative procedure to measure the median and standard deviation of the background over the whole map. For this, we use two tools: sigma-clipping and a segmentation image. Sigma-clipping involves measuring the median and standard deviation of data (in this case pixel values of the SPIRE maps) and removing pixels that are above a clipping limit. The process is then repeated until there are no more pixels above the clipping limit. We chose a clipping limit of 3 standard deviations above the measured median. The function used to perform the sigma-clipping is \texttt{sigma\_clipped\_stats} that is provided within the \texttt{Astropy} \citep{Astropy:2013ek} package. 

Sigma-clipping however can still be affected by sources in the field and provide a biased estimate of the background. A better process is to iteratively run sigma-clipping, each time masking out pixels associated with a source. To determine which pixels will be masked, we use a segmentation image. A segmentation image is a map, the same size as the input map, that identifies groups of connected pixels that are above a certain threshold. For a threshold we use $MD + 2\times SD$ where $MD$ and $SD$ are the median and standard deviation of the map determined through sigma-clipping. A source is identified in the map as a group of 5 interconnected pixels that are above this threshold value. The \texttt{Photutils} function \texttt{detect\_sources} was used to create all segmentation images. All of the pixels that are associated with a source are then masked out and sigma-clipping is re-run on the remaining pixels. This process is repeated until percentage change in the sigma-clipped median is less than $1\times10^{-6}$ or a maximum of 10 iterations. A final sigma-clipped median ($MD_{final}$) and standard deviation ($SD_{final}$) is measured from the masked map.

We then produce a new segmentation image to find the associated BAT source in the SPIRE map using a threshold of $MD_{final} + 1.5\times SD_{final}$. Through tests of various extended sources, we found $1.5\times SD_{final}$ to best incorporate the fainter outer regions of the galaxies. The \texttt{Photutils} function \texttt{segment\_properties} is then used to measure centroid, semimajor axis length, semiminor axis length, and position angle of all sources detected from the segmentation image. We identify the BAT source as the closest detected source within one FWHM (see Section~\ref{sec:obs}) of the known positions (Table 1). 
  
  