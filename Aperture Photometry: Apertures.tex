\subsection{Aperture Photometry}\label{aperture}
\subsubsection{Target and Background Apertures}
For the remaining of 276 sources, aperture photometry was used to measure the flux densities utilizing either the \texttt{annularSkyAperturePhotometry} task within HIPE or the DS9 Funtools tool ``Funcnts'' depending on the size of the source. DS9 and Funcnts were only used for 25 sources where an elliptical aperture was necessary because the source covered a large portion of the image otherwise circular apertures were used with the HIPE task. Aperture sizes were chosen by inspecting the SPIRE images for each source and selecting a radius to fully encompass the FIR emission with minimum radii of 22'', 30'', and 42'', the recommended aperture sizes for point sources from the SPIRE DRG. If a source could not be visually found, then a point source aperture was chosen. 

Annuli with inner and outer radii of 60'' and 90'' were used to estimate the background unless the source aperture had a radius $>60''$, in which case the annulus had a inner radius the same as the source aperture and an outer radius such that it approximately enclosed the same area as a 60--90'' annulus. For sources needing an elliptical aperture, the background elliptical annulus had inner semi major and semi minor axes the same as the source aperture and outer axes 1.5x the inner axes. If bright nearby companions/background sources were present the background annuli were adjusted to avoid contamination. Tables 2 (circular aperture) and 3 (elliptical aperture) list the source aperture and background annulus sizes for each source used in the flux extraction process as well as the center position of the apertures. 6 sources (NGC 2992, NGC 3786, M 106, IC 4329A, NGC 5728, NGC 7465) from Table 3 either were too big to fit a background annulus around the source or had nearby bright companion source(s). For these sources, we used a series of background circular apertures with 22'' or 50'' radius placed around the target source to estimate the background flux density.

Central positions for the apertures were determined using the \texttt{sourceExtractorSussextractor} task within HIPE. This task finds all of the sources in a map within a certain detection threshold which was chosen to be $3\sigma$. The position of the nearest $3\sigma$ source to the NED position (listed in Table 1) of the target source was chosen as the center of the aperture as long as that source was within 1 FWHM length for the specific waveband. If no $3\sigma$ sources were found in the 350 or 500 \um{} band but one existed in the 250 \um{} band then the 250 \um{} position was used. Finally, if the source is undetected in all three wavebands then the NED position is used as the center of the apertures. As a last check, the positions of the apertures were all visually inspected to ensure the correct source's flux density was being measured and the apertures were large enough to cover the entire source.

Some sources, especially those at low galactic latitude, show large smooth, extended emission in the background that is due to cold dust heated by the interstellar radiation field of the Milky Way, called ``cirrus'' emission. Point source aperture sizes were used for all of these sources, as well background annuli much closer to the source rather than the standard 60-90'' annulus. These sources are marked in Table 4 to alert the reader of a possible overestimation of the flux density due to contamination from cirrus.
  